\documentclass[11pt, oneside]{fithesis2}
%PS pro lepsi zarovnani a zlom
\usepackage[protrusion=true,expansion=true]{microtype}
\usepackage{cmap,tgpagella}
\usepackage[T1]{fontenc}
\usepackage[czech]{babel}
\usepackage[utf8]{inputenc}
\usepackage[toc,page]{appendix}
\usepackage{csquotes}
\usepackage{graphicx}
% \usepackage{siunitx}
% \usepackage{pgfplots}
% \usepackage{pgfplotstable} 
\usepackage{filecontents}
\usepackage{booktabs}
\usepackage{etoolbox}
\usepackage{pdfpages}
\usepackage[section]{placeins}
\preto\tabular{\shorthandoff{-}}
\DeclareQuoteStyle{czech}
  {\quotedblbase}
  {\textquotedblleft}
  {\textquoteleft}
  {\textquoteright}

\renewcommand{\uv}[1]{
  \enquote{#1}
}
\usepackage[margin=25pt, labelfont=bf]{caption}
\usepackage{amsmath, amsfonts, pifont, float}
\usepackage[plainpages=false, pdfpagelabels, unicode]{hyperref}
\usepackage[backend=biber, citestyle=numeric]{biblatex}
\usepackage{url}
\setcounter{biburllcpenalty}{7000}
\setcounter{biburlucpenalty}{8000}

%\bibliographystyle{czechiso/czechiso}

\addbibresource{literatura.bib}
\thesistitle{Asistent Posuzovatele}
\thesissubtitle{Diplomová práce}
\thesisstudent{Bc. Adam Bárta}
\thesiswoman{false}
\thesisfaculty{fi}
\thesisyear{2014}
\thesislang{cs}
\thesisadvisor{doc. RNDr. Eva Hladká, Ph.D.}

\begin{document}
\hyphenation{} % sem zadat slova s pomlckami, ktera se nechteji rozdelovat napr. zlu-tou-cky
\FrontMatter
\ThesisTitlePage

% \includepdf{../oficialni_zadani.pdf}
% \includepdf{../prohlaseni_autora.pdf}

\begin{ThesisDeclaration}
\DeclarationText
\AdvisorName
\end{ThesisDeclaration}

\begin{ThesisThanks}
Chtěl bych poděkovat svému školiteli \iffalse doplnte \fi za
cenné rady a~připomínky k~práci, za ochotu při konzultacích a~vstřícné
vedení práce.
\end{ThesisThanks}

\begin{ThesisAbstract}

\end{ThesisAbstract}

\begin{ThesisKeyWords}

\end{ThesisKeyWords}

% vlozit kopie oficialniho zadani a prohlaseni autora skolniho dila, nesmi byt soucasti elektronicke verze (ani jedno)

% vlozit prohlaseni autora skolniho dila
% https://is.muni.cz/auth/do/fi/formulare/SO/Prohlaseni_autora_skolniho_dila_v3.pdf

\MainMatter
\tableofcontents
\chapter{Úvod}

Hlavní cíle této DP -- vytvořit klienta,~převést formuláře do jednotné elektronické podoby,~aplikace nesmí zatěžovat sudí více než je nezbytné.
% \documentclass{article}
% \usepackage[czech]{babel}
% \usepackage[cp1250]{inputenc}

% \begin{document}
\chapter{Základní informace}

\section{Náhled do problematiky}
Při moderních výstavách koček je i~přes velkou dostupnost výpočetní techniky stále upřednostňován zápis výsledků do papírového formuláře.
Ty ale nemají jednotný formát a~jsou často vytištěny pomocí starých jehličkových tiskáren.
To způsobuje,~že i~samotný tisk prázdných formulářů je velice špatný. Jednotliví rozhodčí navíc často informace vpisují nečitelně a~rukou psané hodnocení má rozdílnou úroveň. Problémy dosavadního systému jsou tedy
způsobeny především špatnou čitelností posudků a~v tomto ohledu překonanou metodou ukládání informací.
\linebreak
Problémem budou ale především současná pravidla výstav. Ta při posuzování nedovolují sudím mít žádné elektronické zařízení. Podle aktuálních pravidel by neměl posuzovat například sudí s kardiostimulátorem. --DOPLNIT INFORMACE Z OFICIÁLNÍCH FIFE PRAVIDEL/ODKAZ

\section{Výstavní skupiny}
Aby bylo na výstavách zajištěno přesné rozdělení posuzovaných objektů, jsou kočky děleny do čtyř hlavních skupin podle plemen.
Tyto třídy jsou:
\begin{itemize}
 \item Perské a exotické (Persian \& Exotic)
 \item Polodlouhosrsté (Semilonghair)
 \item Krátkosrsté a somálské (Shorthair \& Somali)
 \item Siamské, orientální a balijské (Balinese, Siamese, Oriental)
\end{itemize}
Hlavní skupiny jsou dále děleny na sedmnáct (resp. osmnáct) soutěžních podtříd\footnote {01 -- supreme šampionů,~02 -- supreme premiorů,~03 -- mezinárodních grandšampionů,~04 -- mezinárodních grandpremiorů,~05 -- mezinárodních šampionů,
~06 -- mezinárodních premiorů,~07 -- šampionů,~08 -- premiorů,~09 -- otevřená nad 10 měsíců,~10 -- kastrátů nad 10 měsíců,~11 -- mladých 6 -- 10 měsíců,~12 -- mladých 3 -- 6 měsíců,~13a -- noviců od 10 měsíců,~13b -- kontrolní třída,
~14 -- domácích koček,~15 -- mimo soutěž,~16 -- vrhy (min. 3 koťata),~17 -- veteránů nad 7 let věku}. Během výstav jsou posuzovatelé zaměřeni na specifické třídy. Během posuzování jsou tedy posuzovatelům přiděleny kočky z různých tříd.

\section{Identifikační systém Easy Mind System}
Easy Mind System\footnote {http://fifeweb.org/wp/breeds/breeds\_ems.php} (dále jen EMS) je jednotný mezinárodní systém federace FIFE
vytvořený pro identifikaci koček za pomoci exteriérových znaků.
V hlavičče výstavního formuláře je uvedený EMS kód složen ze tří částí\footnote{Kompletní popis na http://www1.fifeweb.org/dnld/rules/FIFe\%20EMS\%20system\%20user\%20guidelines.pdf}.
Označení plemena\footnote{To přímo určuje jednu ze čtyř tříd, do keré je daná kočka zařazena}, barvy\footnote{Označené malým písmem} a označení barevného vzoru\footnote{Označeného číslem}.


Při samotném posuzování slouží tedy EMS sudím pro kontrolu vzhledu posuzované kočky v porovnání k datům uvedeným na formuláři.

\section{Formulář}
V současné době se pro zápis výsledků používají pouze papírové formuláře. Posudky jsou zadávány ve třech základních jazycích.
Angličtině, francouzštině, němčině a jazyce země, ve které je výstava pořádána. 

Formát formuláře pro výstavy ale není jednotný. Opakuje se pouze hlavička formuláře, která slouží k identifikaci kočky.
\linebreak

Předvyplněná hlavička formuláře je složena z:
\begin{itemize}
\item Číslo -- unikátní číslo přiřazené kočce na jedné specifické výstavě
\item Plemeno -- rasa kočky
\item EMS System code -- Easy Mind System federace FIFE pro barevné označení kočky
\item Třída -- přiřazená třída, ve které kočka soutěží
\item Pohlaví
\item Datum narození
\end{itemize}

Hlavní část formuláře slouží pro vyplnění jednotlivých částí posudku a výsledného hodnocení.
Posuzování je závyslé na rase kočky. Pro každou rasu jsou stanoveny standardy podle kterých je daný exemplář posuzován.
\linebreak

Posudek je rozdělen na tyto části:
\begin{itemize}
\item Typ
\item Hlava
\item Oči
\item Uši
\item Srst
\item Ocas
\item Kondice
\item Celkový dojem 
\item Komentář
\end{itemize}

\chapter{Analytická část}

\section{Jak výstava probíhá}
V současné podobě probíhá celá výstava téměř bey jakékoliv výpočetní techniky. Seznamy vystavovaných koček a jejich přiřazení k jednotlivým sudím jsou vytištěny v papírové formě. Stejně tak i veškeré informace pro rozhodčí jsou
v tištěné formě.

Po začátku výstavy jsou zveřejněny rozpisy pro jednotlivé sudí a pořadí v kterém jsou zvířata posuzována. Kompletní rozpisy jsou umístěny na dostupném místě. U každého posuyovatelského stolu je navíc vyvěšen rozpis pro konkrétního sudího.
\linebreak

Pořadová čísla zvířat jsou poté vyvolávána pomocí mikrofonu.

Při posuzování samotném sudí nejprve slovně zhodnotí všechny hodnocené aspekty a následně zapíše posudek do formuláře.
Proces poszování je rozdílný sudí od sudího. Neexistuje žádný předepsaný postup posuzování.
\linebreak

Po posouzení jedné skupiny vybírá posuzovatel tzv. Best of Variety (BIV).

Každé plemeno má mezinárodně uznávaný standard. Ten udává ideální vzhled kočky a jeho součástí je bodové ohodnocení jednotlivých hodnocených částí zvířete.
Toto bodové ohodnocení by mělo být uvedeno v hodnotících formulářích. Ve stávajícím stavu toto ohodnocení není ve formuláři uvedeno a posuzovatelé tomuto bodovému ohodnocení nepřikládají velkou váhu.

\section{Zabezpečení}

Komunikace mezi klientem a serverem bude probíhat přes zabezpečenou bezdrátovou komunikaci (Caban 9.3.2). Vlastní hodnocení na výstavách se neobejde bez nervového vypětí. Majitelé koček mají v~mnoha případech silnou motivaci pro 
manipulaci s~výsledky hodnocení ve svůj prospěch. Vhodný výběr zabezpečení je nutnou součástí výstavního systému. 
Omezení pouze výběru MAC adres není vhodné z důvodu velmi snadné prolomitelnosti. Vytvoření seznamu fyzických adres jednotlivých tabletů použitých pro posuzování bude v případě útoku sloužit pouze ke drobnému zdržení.
Dúkladnější útok ale nezadrží. --DOPLNIT PODKLADY! Wi-Fi síť by měla být zabezpečená nejlépe WPA2-PSK. --doplnit délku hesla atd. podle článku!
Otázkou zůstává zabezpečení samotného přenosu dat.

Aplikace (klient) musí nutně zajištovat zálohu dat.
Tzn. po nečekaném výpadku musí dojít k obnovení zadaných dat.
\linebreak

\chapter{Řešení}
Formulář bude převeden do elektronické podoby. Hlavním cílem při přechodu z~dosavadního systému je co největší automatizace doposud ručně prováděných procesů,~s~důrazem na co nejmenší zatížení sudích.
Základní zadaní provedou pořadatelé -- nastavení tabletů a~jejich přiřazení jednotlivým sudím. Tato přidaná práce se odrazí v~efektivnější práci s~formuláři.

Systém se bude starat o~rozdělení koček a~jej ich následné přiřazení posuzovatelům (potažmo spárování s~určitým tabletem). Jakmile se informace o~kočkách zadají do systému,~dojde k~jejich rozdělení a~přiřazení. Do tabletu se nahraje specifická množina koček pro jednoho sudího.
Tato varianta možná přidá práci pořadatelům,~jelikož budou muset dávat dobrý pozor při přidělování zařízení. Jediným rozdílem pro sudí bude tedy nutá kontrola,~zda--li mají v~rukou správný tablet.
Systém se bude také starat o správné umístění posuzovatele. Pokud se jedná například o paralelní posuzování, aplikace se bude chovat totožně jako při klasickém posuzování -- o~zbytek se postará systém sám.
Tento přístup,~kdy sudí nebude zadávat žádné uživatelské jméno a~heslo,~přiblíží elektronické zadávání dosavadnímu papírovému formátu.
Hlavní snahou je nezatěžovat rozhodčí žádnou další aktivitou rozdílnou od vyplňování papírových formulářů. Je třeba odstranit co nejvíce činností spojených s nastavením tabletu, přihlášení do aplikace apod. z povinností rozhodčích.
Vhodnou možností autorizace není ani hardwarový klíč (dongle), tato metoda je blíže popsána v bc. práci Jakuba Cabana v~kapitole 9.1.2..
Jméno posuzovatele se zobrazí například v~pravém horním rohu aplikace 
a použije se při každém uložení hodnocené kočky. Sudí nemá žádnou možnost jak fyzicky podepsat elektronické hodnocení, je tedy třeba jeho jméno doplnit automaticky do každého formuláře.
\linebreak

Rozpoznávání textu -- jedna z možností vylepšení již hotového programu bude doprogramování automatického doplňování slov v posudcích.
Velikost slovníku použitého sudími není nikterak velká. Slova se často opakují a~tudíž by tato schopnost aplikace zřejmě velmi ulehčila a zrychlila průběh samotného posuzování.

\chapter{Implementace}
\section{Použité technologie}
\subsection{ADT Plugin}
Android Developement Tools\footnote{Použitá verze 22.3.0, dostupná na http://developer.android.com/tools/sdk/eclipse-adt.html} (dále jen ADT) je doplněk pro vývojové prostředí Eclipse IDE\footnote{Dostupné na www.eclipse.org}.
ADT obsahuje nástroj pro tvorbu uživatelského rozhraní,~umožňující snadný návrh GUI\footnote {Graphical User Interface -- Grafické uživatelské rozhraní}. Tento soubor nástrojů obsahuje také emulátor zařízení,~díky kterému je možné
vytvořit širokou škálu virtuálních tabletů nebo telefonů založených na operačním systému Android.

\subsection{Android SDK}
Android Software Developement Kit\footnote {Použitá verze 22.3, dostupná na http://developer.android.com/tools/sdk/tools-notes.html} (dále jen SDK) je soubor knihoven a nástrojů nezbytných pro vytváření aplikací pro Android.
Jedná se tedy o~druhý zásadní doplňkek pro Eclipse IDE a programování mobilních aplikací.

\subsection{JSON}
JavaScript Object Notation (dále jen JSON)\footnote {https://code.google.com/p/json-simple/}. Jednotný formát pro předávání dat ve formě strukturovaného textového řetězce. Formát JSON je platformě nezávislý.
Výhodou přenosu dat pomocí JSONu je snadná čitelnost člověkem. 

Hlavní části balíku JSON.simple jsou JSONObject, který slouží k vytváření strukturovaných objektů. JSONParser, který slouží k dělení textového řetězce na jeho podčásti.

\subsection{Android SQLite Database}
Operační systém Android obsahuje vestavěnou SQL databázi\footnote {http://developer.android.com/reference/android/database/sqlite/SQLiteDatabase.html} určenou pro ukládání uživatelských dat.
Tato databáze je primárně určena pro uchování kontaktů,~poznámek a~podobně. Cílem této práce je manipulace s~daty z~formulářů,~které mají předem danou strukturu. Vestavěná SQL databáze se jeví jako vhodné řešení pro ukládání a~manipulování s~těmito daty.

\section{Popis struktury programu pro Android}
Programy pro mobilní systém Android mají rozdílnou strukturu například od programů desktopových prostředí.

Popsat src/ res/ --ten obsahuje XML soubory a pak hlavni soubor AndroidManifest.xml.
Popsat verzi na kterou je moje aplikace cílená, zařízení na kterých bude vyzkoušená a jaká je minimální podporovaná verze.

Klientská aplikace, která je určena pro sudí, má dva hlavní pohledy. Základním pohledem je seznam jednotlivých objektů, které jsou sudímu přiřazeny k posouzení. Po kliknutí na jeden objekt se otevírá okno samotného formuláře.
Tento pohled strukturou kopíruje formát papírového formuláře. Hlavním cílem při návrhu klienta byla co možná nejbližší podoba s původními formuláři. Na rozdíl od papírových formulářů obsahuje elektronická forma ještě specifikace
hodnoceného plemena. Tato specifikace je zobrazena jako bodové ohodnocenéí hodnocených položek pro každé plemeno.

\chapter{Zpracování formulářů}
----TODO - Data od výstavního systému jsou předána klientovi----
Uložení dat do SQL -- Statická data sloužící pro identifikaci kočky. Výběr ohodnocení z SQL databáze probíhá podle zadaného kódu plemena.
\linebreak

V úvodním okně aplikace je sudímu zobrazen list koček, které má přiřazené k posouzení. Po kliknutí na položku listu dochází k zobrazení okna formuláře.
Přídání SQL databáze s jednotlivým bodovým ohodnocením plemen dle standartů FIFE. Jedná se o 100 bodů, přidělených podle důležitosti vnějšího znaku ve vztahu k danému plemenu. V dosavadním hodnocení není na toto bodové ohodnocení
brán zřetel. Přidáním ohodnocení do elektronické formy formuláře by mělo vést ke zkvalitnění a zpřesnění hodnocení.

Android SQL databáze je použita pro ukládání formulářů. V první fázi dochází k zápisu dat ze serveru. Proběhne uložení "hlaviček" formulářů.
V druhé fázi, při samotném vyhodnocování je formulář doplněn sudím, aktualizován v databázi a odeslán zpět na server.

\chapter{Design uživatelského rozhraní}
\section{Ergonomie Softwaru}
Ergonomie je věda zaměřená na zkvalitnění činností a nástrojů tak, aby člověku co nejméně škodily při jejich provozování.
Stejně tak ergonomie softwaru se zabývá zkvalitněním programů.

\section{Uživatelské rozhraní}
Klientská aplikace je cílena na specifickou skupinu uživatelů. Tato skupina je charakteristická svým vyšším věkem a naprostou většinou začátečnických uživatelů.
Důležitými aspekty aplikace jsou tedy jednoduchost, přehlednost a co nejsnazší čitelnost zobrazovaného textu. 

"Začínající uživatelé budou chtít aplikaci co nejjednodušší, aby porozuměli jak ji používat.
Zkušení uživatelé budou chtít aplikaci co nejjednodušší, protože mají jiné věci na práci a nechtějí se zbytečně zdržovat."\cite[str. 43]{hoekman}

Klientská aplikace je rozdělena na dva samostatné logické celky. Ty jsou složeny ze tří hlavních pohledů. Těmito pohledy jsou JurySetting, JuryActivity a MainActivity.

\subsection{JurySettings}
Pohled JurySettings je pohledem sloužícím ke spojení posuzovatelského tabletu se serverem. Jedná se o "pořadatelskou" část klienta, tedy sudí by s tímto pohledem neměl pracovat.
V tomto pohledu jsou dvě kolonky. První je určena pro zadání ip adresy serveru, druhá pro zadání jména sudího kterému bude tablet na dané výstavě přiřazen.
Zadané jméno rozhodčího je zobrazeno v pravém horním rohu aplikace po celou dobu běhu. Jméno nadále slouží pro snadnou identifikaci zařízení.

Pole pro zadání ip adresy povoluje zadat pouze textový řetězec ve správném formátu. Pokud je zadán nepřípustný znak, aplikace uživatele upozorní. --OVERIT VUCI SKUTECNOSTI AZ TO NAPROGRAMUJU!!!--

Po úspěšném spojení serverem a nahrání příslušných dat do tabletu je třeba stisknout tlačítko start. Dojde k ukončení současné aktivity, uzavření socketu a spuštění JurzActivity.
Aplikace se v tomto kroku přepíná z "pořadatelské" do "posuzovatelské" části.
Připravení tabletu pro posuzování obnáší tedy ze strany pořadatelů pouze zadání adresy, jména a dvou kliků.
Pořadatelé předají předpřipravené tablety sudím se spuštěnou aplikací. Sudí obdrží zařízení se zobrazeným listem jemu přiřazených vystavovaných zvířat.

\subsection{JuryActivity}
JuryActivity na počátku zobrazuje seznam všech zvířat, která jsou sudímu přiřazena k posouzení. Kazdá položka tohoto listu zobrazuje data, která identifikují dané zvíře.
V papírovém formuláři jsou tyto informace předtištěny v hlavičce konkrétního formuláře. Po kliknutí na položku seznamu je ukončena aktivita JuryActivity a je spuštěna formulářová aktivita MainActivity.

V tomto bodě sudí přechází k samotnému hodnocení vybraného zvířete.

Pokud je formulář úspěšně vyplněn, změní se barva pozadí dané položky seznamu z počáteční černé na zelenou barvu. Posuzovatel má tedy stále přehled kolik zvířat již ohodnotil a kolik jich ještě zbývá.

Již vyplněné formuláře je možné opakovaně otevřít a upravit. Tato možnost je nutná z důvodu udělování titulů, které probíhá zpětně, až po posouzení všech přidělených koček. 

\subsection{MainActivity}
MainActivity je hlavní formulářovou aktivitou. Formulář je vyvolán rozkliknutím položky seznamu. 
\linebreak

"Designové vzory pomáhají uživatelům v rychlém pochopení nového programu a umožňují jim použít dříve nabyté zkušenosti z jiných zdrojů v nové aplikaci."\cite[str. 172]{hoekman}
\linebreak

Hlavním ukazatelem při návrhu uživatelského rozhraní byl fakt, že většina sudích jsou uživatelé s malou znalostí výpočetní techniky. Z tohoto důvodu podoba elektronického formuláře kopíruje co nejvíce svoji papírovou předlohu.
Uživatelé při vyplňování formulářů tedy nejsou zatěžováni žádnými zbytečnými funkcemi navíc. Mohou proto využít znalosti nabyté při předchozích posuzováních. Mění se pouze forma zadávání informací, vzhled formuláře je zachován.

Hlavička formuláře je automaticky generována z dat obdržených z výstavního systému.
Slouží posuzovateli pro identifikaci zvířete a pro další kontrolu\footnote{Například dle věku zvířete je možné zhistit zda soutěží ve správné třídě.}.

Položky formuláře určené pro vepisování jednotlivých částí posudku jsou opatřeny textovou nápovědou\footnote{Andoroid hint} ve třech hlavních jazycích. Tento text je při vyplňování daného textového pole skyt.
Hodnocení vystavovaného zvířete je možné vzbrat z "roletového menu"\footnote{Android spinner}. Stejně je také řešeno udělování titulu v dané třídě.
\linebreak

Jedinou přidanou hodnotou v tomto pohledu je zobrazování bodového ohodnocení jednotlivých hodnocených aspektů. Toto bodové ohodnocení je uloženo v lokální SQL databázi a liší se pro jednotlivá plemena.
V papírových formulářích toto hodnocení chybí a to i přes to, že by se sudí měli na tyto body brát zřetel.

Zobrazení bodů v elektronickém formuláři má za cíl zpřesnění a usnadnění hodnocení.
\linebreak

Rozebrat možnost, že sudí nebudou brát aplikaci a celý výstavní systém jako klad, ale jako další překážku.
Výstavy probíhají téměř bez jakékoliv automatizace a hlavními nástroji jsou prozatím papír a tužka. Změna zažitého systému musí být provedena pečlivě a s důrazem na cílovou skupinu uživatelů.

Co největší velikost písma (v klientovi nastavane na 30dp) -- snadná čitelnost.


\subsection{Srovnání podobnosti papírového formuláře a elektronického}
--Vložit obrázky naskenovaného listu a screenshotu aplikace--

Navázat kapitolou ergonomie.

\chapter{Java Server}
\section{Server Thread}
--OCITOVAT ZDROJAKY A SNAHU NEOBJEVOVAT KOLO

--$http://www.kieser.net/linux/java_server.html$

Autor: The Kieser.net team.--
\linebreak
Nadefinovat vlastní "protokol" aneb jak si to vlastne povida, doplnit obrazky nebo diagramy komunikace.
\linebreak

Serverové vlákno po spuštění vyčkává na příchozí spojení na definovaném portu s číslem 3344. --Rozepsat se o tom, proč není použito automatické hledání pomocí zasílání UDP packetů na broadcast.
Namísto toho je zadávána adresa serveru na každém klientovi. Poté dochází ke spojení a otevření nového socketu pro každé zařízení. Dopsat maximální počet zařízení (Zatím 10, víc by byla pořádná výstava...)

\section{Client Thread}
Client Thread je navázán na Jury klienta. Při pokusu o spojení se serverem je otevřen socket pro obousměrnou komunikaci.

\section{Práce se sockety}
Serverové vlákno očekává spojení na portu 3344. Pokud klient vzžaduje spojení se serverem, je otevřen socket na tomto portu. Socket je vždy uzavřen po ukončení jedné aktivity klienta (Android activity - okno aplikace).

Při prvotním spuštění aplikace je otevřn pohled JurySettings. Je vyžadováno zadání ip adresy serveru a jméno sudího, kterému bude na aktuální výstavě tablet přiřazen. Socket je otevřen při odeslání
těchto dat a je oetvřen po celou dobu, kdy server odesílá do tabletu hlavičky formulářů. K zavření socketu dochází při ukončení tohoto pohledu.
V tuto chvíli je také spuštěn pohled s výpisem jednotlivých posuzovaných objektů. Sudí zvolí jednu položku a vyplní soutěžní formulář. Při odeslání formuláře je vytvořen nový socket, data jsou uložena do vestavěné SQL databáze
a následně předána serveru jako textový řetězec ve formátu JSON. Poté je aktuální socket uzavřen, aktivita je ukončena a je spuštěn aktualizovaný výpis vzstavovaných zvířat.

\chapter{Problémy navrženého řešení}
\section{Zastaralost pravidel FIFE}
Pro použití klienta a výstavního softwaru na výstavách, bude třeba změnit oficiální pravidla federace FIFE.

Rozebrat celý plánovaný postup nasazení systému v praxi i s problémy a jejich řešením.

\chapter{Google obchod Play - ne}
Nedistribuovat aplikaci přes obchod Play, bude přiložena k výstavnímu systému.

\chapter{Závěr}

\end{document}
