\documentclass[11pt, oneside]{fithesis2}
%PS pro lepsi zarovnani a zlom
\usepackage[protrusion=true,expansion=true]{microtype}
\usepackage{cmap,tgpagella}
\usepackage[T1]{fontenc}
\usepackage[czech]{babel}
\usepackage[utf8]{inputenc}
\usepackage[toc,page]{appendix}
\usepackage{csquotes}
\usepackage{graphicx}
% \usepackage{siunitx}
% \usepackage{pgfplots}
% \usepackage{pgfplotstable} 
\usepackage{filecontents}
\usepackage{booktabs}
\usepackage{etoolbox}
\usepackage{pdfpages}
\usepackage[section]{placeins}
\preto\tabular{\shorthandoff{-}}
\DeclareQuoteStyle{czech}
  {\quotedblbase}
  {\textquotedblleft}
  {\textquoteleft}
  {\textquoteright}

\renewcommand{\uv}[1]{
  \enquote{#1}
}
\usepackage[margin=25pt, labelfont=bf]{caption}
\usepackage{amsmath, amsfonts, pifont, float}
\usepackage[plainpages=false, pdfpagelabels, unicode]{hyperref}
\usepackage[backend=biber, citestyle=numeric]{biblatex}
\usepackage{url}
\setcounter{biburllcpenalty}{7000}
\setcounter{biburlucpenalty}{8000}

% \bibliographystyle{czechiso/czechiso}

\addbibresource{literatura.bib}
\thesistitle{Asistent Posuzovatele}
\thesissubtitle{Diplomová práce}
\thesisstudent{Bc. Adam Bárta}
\thesiswoman{false}
\thesisfaculty{fi}
\thesisyear{2014}
\thesislang{cs}
\thesisadvisor{doc. RNDr. Eva Hladká, Ph.D.}

\begin{document}
\hyphenation{} % sem zadat slova s pomlckami, ktera se nechteji rozdelovat napr. zlu-tou-cky
\FrontMatter
\ThesisTitlePage

% \includepdf{../oficialni_zadani.pdf}
% \includepdf{../prohlaseni_autora.pdf}

\begin{ThesisDeclaration}
\DeclarationText
\AdvisorName
\end{ThesisDeclaration}

\begin{ThesisThanks}
Chtěl bych poděkovat svému školiteli \iffalse doplnte \fi za
cenné rady a~připomínky k~práci, za ochotu při konzultacích a~vstřícné
vedení práce.
\end{ThesisThanks}

\begin{ThesisAbstract}

\end{ThesisAbstract}

\begin{ThesisKeyWords}

\end{ThesisKeyWords}

% vlozit kopie oficialniho zadani a prohlaseni autora skolniho dila, nesmi byt soucasti elektronicke verze (ani jedno)

% vlozit prohlaseni autora skolniho dila
% https://is.muni.cz/auth/do/fi/formulare/SO/Prohlaseni_autora_skolniho_dila_v3.pdf

\MainMatter
\tableofcontents
\chapter{Úvod}

% \documentclass{article}
% \usepackage[czech]{babel}
% \usepackage[cp1250]{inputenc}

% \begin{document}
\chapter{Náhled do problematiky}
Při moderních výstavách koček je i~přes velkou dostupnost výpočetní techniky stále upřednostňován zápis výsledků do papírového formuláře.
Formuláře nemají jednotný formát a~jsou často vytištěny pomocí starých jehličkových tiskáren.
To způsobuje,~že i~samotný tisk prázdných formulářů je velice špatný. Jednotliví soudci/rozhodčí vpisují informace nečitelně. Rukou psané hodnocení má rozdílnou úroveň. Dosavadní systém je problematický především z~důvodu špatné
čitelnosti zápisu.

Hlavní cíle této DP -- vytvořit klienta,~převést formuláře do jednotné elektronické podoby,~aplikace nesmí zatěžovat sudí více než je nezbytné.
\linebreak


Předvyplněná hlavička formuláře
\begin{itemize}
\item Číslo -- unikátní číslo přiřazené kočce na jedné specifické výstavě
\item Plemeno -- rasa kočky
\item EMS System code -- Easy Mind System federace FIFE pro barevné označení kočky
\item Třída -- viz. katalogy výstav koček (rozhodně doplnit do textu)
\item Pohlaví
\item Datum narození
\end{itemize}

Kočka při posuzování dostává bodová ohodnocení,~ale momentálně posuzovatelé tomuto bodovému ohodnocení nepřikládají velkou váhu. Tato možnost bude v~aplikaci zachována.

Posudky jsou psány ve třech oficiálních jazycích: angličtina,~němčina,~francouzština a~nebo v~jazyce země,~kde se výstava pořádá.

Formulář bude převeden do elektronické podoby. Hlavním cílem při přechodu z~dosavadního systému je co největší automatizace doposud ručně prováděných procesů,~s~důrazem na co nejmenší zatížení sudích.
Základní zadaní provedou pořadatelé -- nastavení tabletů a~jejich přiřazení jednotlivým sudím. Tato přidaná práce se odrazí v~efektivnější práci s~formuláři.
\linebreak

Položky formuláře
\begin{itemize}
\item Typ
\item Hlava
\item Oči
\item Uši
\item Srst
\item Ocas
\item Kondice
\item Celkový dojem 
\item Komentář
\end{itemize}

Systém se bude starat o~rozdělení koček a~jejich následné přiřazení posuzovatelům (potažmo spárování s~určitým tabletem). Jakmile se informace o~kočkách zadají do systému,~dojde k~jejich rozdělení a~přiřazení. Do tabletu se nahraje specifická množina koček pro jednoho sudího.
Tato varianta možná přidá práci pořadatelům,~jelikož budou muset dávat dobrý pozor při přidělování zařízení. Jediným rozdílem pro sudí bude tedy nutá kontrola,~zda--li mají v~rukou správný tablet.
Systém se bude také starat o správné umístění posuzovatele. Pokud se jedná například o paralelní posuzování, aplikace se bude chovat totožně jako při klasickém posuzování -- o~zbytek se postará systém sám.
Tento přístup,~kdy sudí nebude zadávat žádné uživatelské jméno a~heslo,~přiblíží elektronické zadávání dosavadnímu papírovému formátu.
Hlavní snahou je nezatěžovat rozhodčí žádnou další aktivitou rozdílnou od vyplňování papírových formulářů. Je třeba odstranit co nejvíce činností spojených s nastavením tabletu, přihlášení do aplikace apod. z povinností rozhodčích.
Vhodnou možností autorizace není ani hardwarový klíč (dongle), tato metoda je blíže popsána v bc. práci Jakuba Cabana v~kapitole 9.1.2..
Jméno posuzovatele se zobrazí například v~pravém horním rohu aplikace 
a použije se při každém uložení hodnocené kočky. Sudí nemá žádnou možnost jak fyzicky podepsat elektronické hodnocení, je tedy třeba jeho jméno doplnit automaticky do každého formuláře.
\linebreak

Rozpoznávání textu -- jedna z možností vylepšení již hotového programu bude doprogramování automatického doplňování slov v posudcích.
Velikost slovníku použitého sudími není nikterak velká. Slova se často opakují a~tudíž by tato schopnost aplikace zřejmě velmi ulehčila a zrychlila průběh samotného posuzování.

\chapter{Zabezpečení}

Komunikace mezi klientem a serverem bude probíhat přes zabezpečenou bezdrátovou komunikaci (Caban 9.3.2) nejlépe s nastaveným filtrováním MAC adres.
Filtrování fyzických adres případného útočníka zajisté nezadrží,~ale minimálně zpomalí.
Otázkou zůstává zabezpečení samotného přenosu dat.

Aplikace (klient) musí nutně zajištovat zálohu dat.
Tzn. po nečekaném výpadku musí dojít k obnovení zadaných dat.\footnote{Použít zasílání dat ve formá logů? Co se stane když se nezazálohuje jedna právě vyplňovaná kolonka formuláře, ale zbytek se obnoví?}
\linebreak

\chapter{Analytická část}
\section{Jak výstava probíhá}



\chapter{Řešení}

\chapter{Implementace - Použité technologie}
\section{ADT Plugin}
Android Developement Tools \footnote{Použitá verze 22.3.0, dostupná naa http://developer.android.com/tools/sdk/eclipse-adt.html} (dále jen ADT) je doplněk pro vývojové prostředí Eclipse IDE. \footnote{Dostupné na www.eclipse.org}
Tento soubor nástrojů obsahuje také emulátor zařízení,~umožnující vytvořit širokou škálu virtuálních tabletů nebo telefonů založených na operačním systému Android.
Tento emulátor je použit ke zkoušení aplikace na několika různých tabletech, pro zajištění co nejvetší kompatibility s různými zařízeními.

Druhým potřebným doplňkem pro instalaci ADT a pro programování mobilních aplikací je tzv. SDK Tools \footnote {Použitá verze 22.3, dostupná na http://developer.android.com/tools/sdk/tools-notes.html}

\section{Android SDK}
Android Software Developement Kit (dále jen SDK)

\section{JSON}
JavaScript Object Notation (dále jen JSON). Jednotný formát pro předávání dat, který je platformě nezávislý. Výhodou přenosu dat pomocí JSONu je snadná čitelnost člověkem. 

\chapter{Popis struktury programu pro Android}
Programy pro mobilní systém Android mají rozdílnou strukturu například od programů desktopových prostředí.

Popsat src/ res/ --ten obsahuje XML soubory a pak hlavni soubor AndroidManifest.xml.
Popsat verzi na kterou je moje aplikace cílená, zařízení na kterých bude vyzkoušená a jaká je minimální podporovaná verze.

\chapter{Ergonomie Softwaru}
Vysvětlit strohost klienta. Vyzvednout klad jednoduchosti.
\linebreak

Ergonomie je věda zaměřená na zkvalitnění činností a nástrojů tak, aby člověku co nejméně škodily při jejich provozování.
Stejně tak ergonomie softwaru se zabývá zkvalitněním programů.
\linebreak

Aplikace je cíleně navržena pro specifickou skupinu lidí. Sudí na výstavách jsou vyššího věku. Dá se předpokládat že naprostá většina z nich nemá zkušenosti s operačním systémem android ani s mobilními aplikacemi.

Rozebrat možnost, že sudí nebudou brát aplikaci a celý výstavní systém jako klad, ale jako další překážku.
Výstavy probíhají téměř bez jakékoliv automatizace a hlavními nástroji jsou prozatím papír a tužka. Změna zažitého systému musí být provedena pečlivě a s důrazem na cílovou skupinu uživatelů.

Aplikace by měla vyhledově co nejpřesněji kopírovat papírové formuláře, aby nedošlo ke zmatení uživatelů.

\end{document}
